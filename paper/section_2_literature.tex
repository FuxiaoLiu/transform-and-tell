% !TEX root = main.tex

\section{Related Works}

Early image captioning systems mainly used an RNN architecture with a closed
vocabulary \cite{Karpathy2015DeepVA, Donahue2015LongTR, Vinyals2015ShowAT}.
Attention over image patches was later introduced in ``Show, Attend and Tell"
\cite{Xu2015ShowAA}, in which the attention weights are obtained by feeding the
image embeddings and the previous hidden state of the RNN through a multilayer
perception. Later improvements include giving the model the option to not
attend to any image region \cite{Lu2017KnowingWT}, using reinforcement learning
to directly optimize for the CIDEr metric \cite{Rennie2017SelfCriticalST,
Gao2019DeliberateAN}, and using a bottom-up approach to propose a region to
attend to \cite{Anderson2017BottomUpAT}. All of these systems can only generate
generic captions with a limited vocabulary.

A more challenging task is to produce captions for news images, in which we
need to be able to generate entity names. Early non-neural approaches include
extractive methods that use n-gram models to combine existing phrases
\cite{Feng2013AutomaticCG} or simply retrieving the most representative
sentence \cite{Tariq2017ACE} in the article. Ramisa \etal
\cite{Ramisa2016BreakingNewsAA} concatenated the word2vec representation of the
article and and the VGG19 representation of the image, and feed them as inputs
to an LSTM generator. However the generator still cannot produce names not
seen in training.

To overcome the limitation of a fixed-size vocabulary, template-based methods
have been used to insert named entities. This involves first generating a
template sentence with placeholders, e.g. ``PERSON speaks at BUILDING in
DATE.'' Afterwards, a selection algorithm is used to pick the best candidate
for each placeholder. Lu \etal \cite{Lu2018EntityAI} built a knowledge graph
for each combination of entities and select the most likely combination.
Meanwhile Biten \etal~\cite{Biten2019GoodNews}, whose GoodNews dataset we will
benchmark against, picked the sentence with the highest cosine similarity with
the template and then found the first entity that matches the type of each
placeholder for insertion. Our proposed model differs from
\cite{Lu2018EntityAI} and \cite{Biten2019GoodNews} in that we are able to
generate a caption with named entities directly without using any intermediate
template.

\begin{figure*}[t]
    \begin{center}
    \fbox{\rule{0pt}{2in} \rule{.9\linewidth}{0pt}}
    \end{center}
       \caption{Overall architecture of the model.}
    \label{fig:short}
 \end{figure*}


Transformers are still scarcely used in image captioning. They have been shown
to yield competitive results in generating generic MS COCO captions
\cite{Zhu2018CaptioningTW, Li2019Boosted}. Zhao \etal
\cite{Zhao2019InformativeIC} have gone further and trained transformers to
produce named entities in the Conceptual Captions dataset
\cite{Sharma2018ConceptualCA}. However Conceptual Captions have no additional
context apart from the image itself, and the authors used web-entity labels,
extracted using Google Cloud Vision API, as inputs to the model. In our work,
we are more ambitious in that we do not explicitly give the model a list of
entities that should appear in the caption. Instead the model has to determine
on its own which entities to generate by scanning through and attending to the
article.

BPE offers an elegant solution to handling an open vocabulary. To date the only
image captioning work that uses BPE is \cite{Sharma2018ConceptualCA}, but in
their data preprocessing step, they explicitly removed rare named entities from
the captions. We attempt to fill this gap and in particular examine how much
better BPE can generate rare names compared to template-based methods.

In addition to attending to image patches, some captioning models also attend
to object regions \cite{Wang2019Hierarchical} and visual concepts
\cite{You2016ImageCW,Li2019Boosted,Wang2019Hierarchical}, both of which are
derived from the image itself. When attending to more than one modality, there
are various strategies on how to combine embeddings such as addition,
concatenation, and using multivariate residual modules (MRMs)
\cite{Kim2016MultimodalRL}. In our models, we use a simple concatenation since
more complex strategies such as MRMs have shown to yield only minor improvement
\cite{Wang2019Hierarchical}.
