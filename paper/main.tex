\documentclass[10pt,twocolumn,letterpaper]{article}

\usepackage{cvpr}
\usepackage{times}
\usepackage{epsfig}
\usepackage{graphicx}
\usepackage{multirow} % creates cells that span multiple rows in a table
\usepackage{amsmath}
\usepackage{amssymb}
\usepackage{bm} % bold fonts for math symbols
\usepackage{booktabs}
\usepackage[export]{adjustbox} % adds a frame around figures
\usepackage{IEEEtrantools} % IEEEeqnarray math equation environment
\usepackage{multirow} % merge multiple rows
\usepackage{tabularx} % for 'tabularx' environment
\usepackage{caption} % controls spacing between caption and table

% Include other packages here, before hyperref.

% If you comment hyperref and then uncomment it, you should delete
% egpaper.aux before re-running latex.  (Or just hit 'q' on the first latex
% run, let it finish, and you should be clear).
\usepackage[pagebackref=true,breaklinks=true,letterpaper=true,colorlinks,bookmarks=false]{hyperref}

\newcommand{\eat}[1]{{}}

\newcommand{\rev}[1]{{\color{blue}{#1}}}
\newcommand{\verify}[1]{{\color{red}{#1}}}

% \cvprfinalcopy % *** Uncomment this line for the final submission

\def\cvprPaperID{7245} % *** Enter the CVPR Paper ID here
\def\httilde{\mbox{\tt\raisebox{-.5ex}{\symbol{126}}}}

% Pages are numbered in submission mode, and unnumbered in camera-ready
\ifcvprfinal\pagestyle{empty}\fi
\begin{document}

%%%%%%%%% TITLE - suggestions and candidates
\title{Transform and Tell:\\The News Image Caption Writer That Remembers Rare Names}
\title{Transforming Entities for News Image Captions}

\author{Alasdair Tran, Alexander P. Matthews, Lexing Xie\\
Australian National University\\
{\tt\small \{alasdair.tran,alex.matthews,lexing.xie\}@anu.edu.au}
% For a paper whose authors are all at the same institution,
% omit the following lines up until the closing ``}''.
% Additional authors and addresses can be added with ``\and'',
% just like the second author.
% To save space, use either the email address or home page, not both
}

\maketitle
%\thispagestyle{empty}

% !TEX root = main.tex

%%%%%%%%% ABSTRACT
\begin{abstract}
    We propose an end-to-end model to generate image captions in news articles.
    By combining the transformer architecture, byte-pair encoding, and
    pretrained embeddings from three different modalities (RoBERTa for text,
    ResNet-152 for images, and FaceNet for faces), our system is able to
    describe an image with specific named entities mentioned in the article. Our
    model achieves a CIDEr score of 54 on the GoodNews dataset, significantly
    outperforming the previous state-of-art CIDEr of 13. We also introduce the
    NYTimes800k dataset, the largest news image captioning dataset to date.
    NYTimes800k is an extended version of GoodNews with higher-quality articles and
    metadata that allow us to study the importance of the image location within
    the text. On NYTimes800k, we achieve a CIDEr of 55. Pretrained models and
    source code are available from
    \href{https://github.com}{https://github.com/anonymized-link}.
 \end{abstract}

% !TEX root = main.tex

\section{Introduction}



The internet is home to a huge number of images, many of which lack useful
captions. A growing body of work seeks to automatically generate captions that
describe the objects and relationships using only visual cues extracted from
the image itself~\cite{Donahue2015LongTR, Vinyals2015ShowAT, Fang2015FromCT,
Karpathy2015DeepVA, Rennie2017SelfCriticalST, Lu2017KnowingWT,
Anderson2017BottomUpAT, Cornia2019ShowCT}. While generic image descriptions
have their uses, such as for individuals with vision impairments, they are
often of less benefit to the average user. To produce more useful image
captions we need to go beyond generic descriptions and introduce information
that cannot be gleaned directly from the image alone. Fortunately, many images
have an associated context such as a news article, web page, or social
media post, which give the image greater meaning than can be extracted from its
pixels. To generate captions that go beyond generic description and actually
add information that could not be gleaned from the image alone we must take
this context into account. We focus on the news image captioning task in order
to design practical methods for exploiting contextual information.

\eat{However, images on the internet are
often associated with a context such as a news article, web page, or social
media post, which give the image greater meaning than can be extracted from its
pixels. To generate captions that go beyond generic description and actually
add information that could not be gleaned from the image alone we must take
this context into account.}

News image captioning is an interesting instance of contextual captioning
where news articles provide context to images.

Captions for news images, such as the example in Figure~\ref{fig:teaser},
typically contain details which cannot be derived from the image alone. They
also frequently contain proper nouns such as names of people, places, and
organizations -- in many cases these proper nouns are rare (most people and
places do not have many news articles written about them). A system capable of
generating high quality news captions should therefor make extensive use of the
provided context and be tuned for generating rare proper nouns. Existing
approaches to news image captioning~\cite{Tariq2017ACE,
Ramisa2016BreakingNewsAA,
	Biten2019GoodNews}  rely on text extraction or
template filling to deal with rare contextual terms such as names of people and
organizations. This makes them relatively inflexibility and means they cannot
be trained end-to-end. Moreover, existing approaches do not include specialized
visual models for frequent nouns -- experiments on the MSCOCO dataset have
shown
that pre-trained object detectors tuned for frequent nouns
lead to more accurate captions~\cite{}.

\begin{figure}[t]
	\begin{center}
		\includegraphics[width=0.99\linewidth, frame]{figures/figure_1.pdf}
	\end{center}
	\caption{Our transformer model attends to embeddings from three different
		domains (image patches, faces, and article text). Using byte-pair
		encoding, the model can then directly produce a caption containing
		specific named entities without the use of templates.}
	\label{fig:long}
	\label{fig:teaser}
\end{figure}



\eat{The contextual word
	embeddings provided by these methods have helped establish new
	state-of-the-art
	results on many natural language understanding benchmarks including GLUE
	\cite{Wang2019GLUE}, SuperGLUE \cite{Wang2019SuperGLUEAS}, and SQuAD
	\cite{Rajpurkar2016SQuAD, Rajpurkar2018KnowWY}, even surpassing the human
	baselines in many cases.}

\eat{In parallel to the development of these pre-training methods, many novel
techniques have been proposed to improve training convergence and handle more
diverse data inputs. The most significant contribution is the use of byte-pair
encoding (BPE) \cite{Sennrich2015NeuralMT} to represent a rare word as a
sequence of subword units, thus giving models the ability to handle an open
vocabulary.}




This motivates our novel fully end-to-end
model for news image captioning that 1) combines specialised modules for
incorporating and selectively attending to image features, human faces, and
news article text and
2) applies a state-of-the-art sequence generation model which is able to
generate rare tokens, such as proper names, even when they do not form part of
the
training data. Our model relies on a novel combination of sequence-to-sequence
architectures, language representation learning, and
vision systems.

%The key components of our approach are selectively attending to different
%aspects of the two modalities: image
%regions, faces, and text sequences.
%Using powerful pre-trained contextual language models to represent news
%articles -- allowing articles with text not see during training  tuples
%required for training.
%Using sub-word units, which when combined with pre-training, allow for the
%generation of tokens not seen during training.


%\sout{
%Most existing captioning systems can only generate a generic description
%of an
%image \cite{Donahue2015LongTR, Vinyals2015ShowAT, Fang2015FromCT,
%Karpathy2015DeepVA, Rennie2017SelfCriticalST, Lu2017KnowingWT,
%Anderson2017BottomUpAT, Cornia2019ShowCT}, mainly because these architectures
%cannot handle an open vocabulary and early captioning datasets such as MS COCO
%\cite{Lin2014MicrosoftCC, Chen2015MicrosoftCC} and Flickr30k
%\cite{Young2014FromID} were annotated using only visual cues present in the
%image. More recently, datasets such as TIME \cite{Tariq2017ACE}, BreakingNews
%\cite{Ramisa2016BreakingNewsAA}, and GoodNews \cite{Biten2019GoodNews} include
%real-life captions written by professional journalists. However, existing
%models that are trained on these news datasets still need to rely on either
%extractive methods or templates to insert named entities.}




In this paper we carefully consider the news image captioning problem and
select a set of modeling tools which we combine into a novel architecture that
sets a new state-of-the-art result. Our main contributions are threefold:

\eat{The goal of this paper is to make use of these recent developments, along
with
their bags of tricks, to carefully design a captioning system most appropriate
for news images.}

\begin{enumerate}
   \item We introduce NYTimes800k, the largest news image captioning dataset to
   date, containing 446K articles and 794K images with captions from The New
   York Times spanning 14 years. NYTimes800k builds on the GoodNews dataset;
   but we write a custom parser to collect higher-quality articles and metadata
   such as the location of an image within the page.

   \item We build a captioning model that combines the power of transformers,
   byte-pair encoding, copying via multi-headed attention, and attention over
   three different modalities (text, images, and faces). We show that our model
   achieves state-of-the-art results with a significant margin over previous
   methods, and in particular, it can generate names not seen during training
   without the use of templates.

   \item We provide a detailed model analysis, deconstructing the most
   important modeling components and quantifying the incremental contribution
   that each of them makes not only to the usual metrics such as BLEU, ROUGE,
   METEOR, and CIDEr; but also to other linguistic measures like readability
   scores, caption length, and recall of rare names.
\end{enumerate}

% !TEX root = main.tex

\section{Related Works}

A large number of methods exist for generating generic image captions that
describe objects and relationships using only image information. Many of these
captioning systems use some combination of a Convolutional Neural Network
encoder and a RNN with a closed
vocabulary as a decoder~\cite{Karpathy2015DeepVA, Donahue2015LongTR,
Vinyals2015ShowAT}.
Attention over image patches was introduced in ``Show, Attend and Tell"
\cite{Xu2015ShowAA}, in which the attention weights are obtained by feeding the
image embeddings and the previous hidden state of the RNN through a multilayer
perception. Many extensions to these models have been proposed such as giving
the model the option to not
attend to any image region~\cite{Lu2017KnowingWT}, using reinforcement learning
to directly optimise for the CIDEr metric~\cite{Rennie2017SelfCriticalST,
Gao2019DeliberateAN}, and using a bottom-up approach to propose a region to
attend to \cite{Anderson2017BottomUpAT}. All of these systems generate
restricted vocabulary generic captions without considering context external to
the image.

A related task which does consider image context is news image captioning,
where the image caption is generated using the article text as context.
One key challenge of news image captioning is generating rare entity names, for
example the names of people who do not make many media appearances. Early
non-neural approaches include
extractive methods that use n-gram models to combine existing phrases
\cite{Feng2013AutomaticCG} or simply retrieving the most representative
sentence \cite{Tariq2017ACE} in the article. Ramisa \etal
\cite{Ramisa2016BreakingNewsAA} concatenated the word2vec representation of the
article and and the VGG19 representation of the image, and feed them as inputs
to an LSTM generator. However the generator still cannot produce names not
seen in training.

To overcome the limitation of a fixed-size vocabulary, template-based methods
have been used to insert named entities. This involves first generating a
template sentence with placeholders, e.g. ``PERSON speaks at BUILDING in
DATE.'' Afterwards, a selection algorithm is used to pick the best candidate
for each placeholder. Lu \etal \cite{Lu2018EntityAI} built a knowledge graph
for each combination of entities and select the most likely combination.
Meanwhile Biten \etal~\cite{Biten2019GoodNews}, whose GoodNews dataset we will
benchmark against, picked the sentence with the highest cosine similarity with
the template and then found the first entity that matches the type of each
placeholder for insertion. Our proposed model differs from
\cite{Lu2018EntityAI} and \cite{Biten2019GoodNews} in that we are able to
generate a caption with named entities directly without using any intermediate
template.

\begin{figure*}[t]
    \begin{center}
    \fbox{\rule{0pt}{2in} \rule{.9\linewidth}{0pt}}
    \end{center}
       \caption{Overall architecture of the model.}
    \label{fig:short}
 \end{figure*}


Transformers are still scarcely used in image captioning. They have been shown
to yield competitive results in generating generic MS COCO captions
\cite{Zhu2018CaptioningTW, Li2019Boosted}. Zhao \etal
\cite{Zhao2019InformativeIC} have gone further and trained transformers to
produce named entities in the Conceptual Captions dataset
\cite{Sharma2018ConceptualCA}. However Conceptual Captions have no additional
context apart from the image itself, and the authors used web-entity labels,
extracted using Google Cloud Vision API, as inputs to the model. In our work,
we are more ambitious in that we do not explicitly give the model a list of
entities that should appear in the caption. Instead the model has to determine
on its own which entities to generate by scanning through and attending to the
article.

BPE offers an elegant solution to handling an open vocabulary. To date the only
image captioning work that uses BPE is \cite{Sharma2018ConceptualCA}, but in
their data preprocessing step, they explicitly removed rare named entities from
the captions. We attempt to fill this gap and in particular examine how much
better BPE can generate rare names compared to template-based methods.

In addition to attending to image patches, some captioning models also attend
to object regions \cite{Wang2019Hierarchical} and visual concepts
\cite{You2016ImageCW,Li2019Boosted,Wang2019Hierarchical}, both of which are
derived from the image itself. When attending to more than one modality, there
are various strategies on how to combine embeddings such as addition,
concatenation, and using multivariate residual modules (MRMs)
\cite{Kim2016MultimodalRL}. In our models, we use a simple concatenation since
more complex strategies such as MRMs have shown to yield only minor improvement
\cite{Wang2019Hierarchical}.

% !TEX root = main.tex

\section{Model Architecture}

\subsection{Encoders}

Our proposed model takes three inputs: the image, the faces, and article text.
Each of these inputs first goes through an encoder to give us a vector
representation.

\subsubsection{Image Embedding}

For the image, we feed it through a pretrained ResNet-152 model
\cite{He2016ResNet} and use the output of the final block, just before the
pooling layer, as the image embedding $\bX_I \in \mathbb{R}^{2048 \times 49}$.
The embedding forms a 7 by 7 block, allowing the transformer to attend to 49
different patches in the image.

\subsubsection{Article Embedding}

For the article, we use RoBERTa \cite{Liu2019RoBERTaAR} to encode the text.
RoBERTa, a more carefully trained BERT \cite{Devlin2019BERT}, is a language
representation model that provides pretrained contextual embeddings for text.
Unlike GloVe \cite{Pennington2014Glove} and word2vec
\cite{Mikolov2013DistributedRO} embeddings, where each word has exactly one
representation, the bidirectionality and the attention mechanism in the
transformer allow a word to have different vector representations depending on
the surrounding context.

The largest GloVe model has a vocabulary size of 1.2 million. Although this is
large, many rare names will still get mapped to the unknown token. In contrast,
RoBERTa uses BPE \cite{Sennrich2015NeuralMT,Radford2019LanguageMA} which can
encode any word that can be written in Unicode characters.

One limitation of RoBERTa is that the maximum length of the input sequence is
512. For GoodNews, we simply encode the first 512 tokens of the article. For
NYTimes800k, since we have the image position, we concatenate the title, the
first paragraph, and as many paragraphs above and below the image as we can
fit, until we reach the 512 token limit. Note that since we are using BPE, a
word might consist of many tokens. On average, we can only encode ...... words
of the article.

The RoBERTa encoder provides gives us the article embedding $\bX_T \in
\mathbb{R}^{1024 \times S}$ where $S$ is the number of tokens.


\subsubsection{Face Embedding}

To embed the faces, we first use MTCNN \cite{Zhang2016JointFD} to detect the
face bounding boxes. We then select the top $M$ faces and feed them through a
FaceNet model \cite{Schroff2015FaceNetAU}, pretrained on the VGGFace2 dataset
\cite{Cao2017VGGFace2AD}, to obtain a face embedding $\bX_F \in
\mathbb{R}^{512 \times M}$ where $M$ is the number of faces.

\subsection{Decoder}

The decoder is a function that estimates $p(y_t)$, the probability of the next
token, conditional on the past $\by_{<t}$ and the context embeddings $\bX_I$,
$\bX_T$, and $\bX_F$:
\begin{IEEEeqnarray*}{lCl}
   p(y_t) &=& \mathbb{P}(Y_t = y_t \mid \by_{<t}, \bX_I, \bX_T, \bX_F)
\end{IEEEeqnarray*}
In our architecture, the decoder consists of four layers of transformer blocks.
In each block, the conditioning on past tokens is computed using dynamic
convolutions \cite{Wu2018PayLA}, and the conditioning on the contexts is
computed using multi-head attention \cite{Vaswani2017AttentionIA}.


\subsubsection{Multi-Head Attention}

Let $\bh_t \in \mathbb{R}^{1024}$ be the current hidden state when decoding the
$t$th token. We divide the embedding dimension into 16 attention heads, each
with size 64. For each head $i \in \{1, 2, ..., 16\}$, we first do a linear
projection of $\bh_t$ and the image embedding $\bX_I$ into a query $\bq_{It}
\in \mathbb{R}^{64}$, key $\bK_I \in \mathbb{R}^{64 \times 49}$, and value
$\bV_I \in \mathbb{R}^{64 \times 49}$:
\begin{IEEEeqnarray*}{lCl}
   \bq_{Iit} &=& \bW_{Iiq} \, \bh_t \\
   \bK_{Ii} &=& \bW_{Iik} \, \bX_I \\
   \bV_{Ii} &=& \bW_{Iiv} \, \bX_I
\end{IEEEeqnarray*}
Then the attended image for each head $\bx'_{Ii} \in \mathbb{R}^{64}$ is the
weighted sum of the values, where the weights are obtained from the dot product
between the query and key:
\begin{IEEEeqnarray*}{lCl}
   \bx'_{Ii} &=& \text{softmax}\left(\bK_{Ii}^T \bq_{Iit} \right) \bV_{Ii}
\end{IEEEeqnarray*}
The attention from each head is then concatenated into $\bx'_{I} \in
\mathbb{R}^{1024}$:
\begin{IEEEeqnarray*}{lCl}
   \bx'_{I} &=& [\tx_{I1}, \tx_{I2}, ..., \tx_{I16}]
\end{IEEEeqnarray*}
and the overall image attention $\tx_{I} \in \mathbb{R}^{1024}$ is obtained
after adding a residual connection and layer normalization:
\begin{IEEEeqnarray*}{lCl}
   \tx_{I} &=& \text{LayerNorm}(\bh_t + \bx'_{I})
\end{IEEEeqnarray*}
We use the same attention mechanism (with different weight matrices) to obtain
the attended article $\tx_T$ and the attended face $\tx_F$. These three are
then concatenated and fed through a feedforward layer:
\begin{IEEEeqnarray*}{lCl}
   \tx_C &=& [\tx_I, \tx_T, \tx_F] \\
   \tx_R &=& \bW_C \, \tx_C + \bb_C \\
   \tx_D &=& \text{ReLU}(\bW_R \, \tx_R + \bb_R )\\
   \tx_E &=& \text{LayerNorm}(\tx_R + \bW_D \, \tx_D + \bb_D)
\end{IEEEeqnarray*}
The final output $\tx_E \in \mathbb{R}^{1024}$ is used as the input to the
next transformer block.

\subsubsection{Dynamic Convolutions}

Instead of using the standard self-attention module as in current
state-of-the-art GPT-2 architecture \cite{Radford2019LanguageMA}, we find that
dynamic convolutions \cite{Wu2018PayLA} are more efficient to train, especially
when there is limited GPU compute power.

Given an input $\bz_t \in \mathbb{R}^{1024}$, which can either be the input
embedding of the previous token, or the output from the previous transformer
block, we first use a feedforward layer with a gated linear unit (GLU)
\cite{Dauphin2017GLU} as the activation function:
\begin{IEEEeqnarray*}{lCl}
   \bz_t' &=& \text{GLU}(\bW_Z \, \bz_t + \bb_Z)
\end{IEEEeqnarray*}


\subsection{Bag of Tricks}

\subsubsection{Mixing RoBERTa layers}
RoBERTa consists of 24 layers of bidirectional transformer blocks. Given an
input of length $S$, the pretrained RoBERTa encoder will return 25 sequences of
embeddings, $\bG_i \in \mathbb{R}^{2048 \times S}$ for $i \in \{0,1,
2,...,24\}$. This includes the initial uncontextualized embeddings and the
output of each of the 24 layers. Inspired by Tenney \etal
\cite{Tenney2019BertRT}, who showed that different layers in BERT represent
different steps in the traditional NLP pipeline, we take a weighted sum across
all layers to obtain the article embedding:
\begin{IEEEeqnarray*}{lCl}
   \bX_T &=& \sum_{i=0}^{24} \alpha_i \bG_i
\end{IEEEeqnarray*}
where $\bx_T \in \mathbb{R}^{1024 \times S}$ is the article embedding
and $\alpha_i$ are learnable weights.

\subsubsection{Copying with Multi-headed Attention}

Inspired by pointer-generator networks \cite{See2017GetTT}, we introduce
a copying mechanism using multi-headed attention.

\subsubsection{Adaptive Softmax}

The BPE vocabulary size is 50265. To make training more efficient, we use adaptive softmax
\cite{Grave2016EfficientSA} and divide the vocabulary into three clusters: 5K,
15K, and 25K. We tie the adaptive weights and we share the decoder input and
output embeddings. We use sinusoidal positional encoding
\cite{Vaswani2017AttentionIA} to represent the position of each token.

% !TEX root = main.tex

% \begin{table}[t]
% 	\caption {Summary of datasets}
% 	\label{tab:datasets}
% 	\centering
% 	\begin{tabular}{llll}
% 		\toprule
% 		  & GoodNews  & GoodNews+ &   NYTimes800k \\
% 		\midrule
%       No. of articles & 241 808 & 241 808 & 445 828 \\
%       No. of images   & 462 642 & 440 112 & 794 085 \\
%       Article length & 451 & 963 & 974 \\
%       Caption length & 18 & 18 & 18 \\
%       Start month & Jan 10 & Jan 10 & Mar 05\\
%       End month & Jul 18 & Jul 18 & Sep 19 \\
% 		\bottomrule
% 	\end{tabular}
% \end{table}


\begin{table}[t]
	\caption {Summary of news captioning datasets}
	\label{tab:datasets}
	\centering
	\begin{tabularx}{\linewidth}{lXX}
		\toprule
		  & GoodNews  &   NYTimes800k \\
		\midrule
      Number of articles & 257 033 & 445 819 \\
      Number of images   & 462 642 & 794 044 \\
      Average article length & 451 & 974 \\
      Average caption length & 18 & 18 \\
      Collection start month & Jan 10 & Mar 05\\
      Collection end month & Mar 18 & Sep 19 \\
      \midrule
      \% of words that are \\
      \quad -- nouns & 16\% & 16\% \\
      \quad -- pronouns & 1\% & 1\% \\
      \quad -- proper nouns & 23\% & 22\% \\
      \quad -- verbs & 9\% & 9\%  \\
      \quad -- adjectives & 4\% & 4\% \\
      \quad -- named entities & 27\% & 26\% \\
      \quad -- personal names & 9\% & 9\% \\
      \midrule
      \% of captions with \\
      \quad -- named entities & 97\% & 96\% \\
      \quad -- personal names & 68\% & 68\% \\
		\bottomrule
	\end{tabularx}
\end{table}

\section{Datasets}

\subsection{GoodNews}

To compare to existing approaches we use the GoodNews dataset, which until now
was largest
dataset for news image captioning~\cite{Biten2019GoodNews}. Each example in the
dataset is a triplet containing an article, an image, and a caption. Since only
the article text, captions, and image URLs are publicly released
the images need to be downloaded from the original source. Out of the 466K
image
URLs provided by
\cite{Biten2019GoodNews}, we were able to download 463K images, or 99.2\% of the
original dataset -- the remaining are broken links.

We use this 99.2\% sample of the GoodNews dataset and the
train-validation-test split provided by~\cite{Biten2019GoodNews}. There are
421K training, 18K validation, and 23K test captions. Note that
this split was performed at the level of captions, so it is possible for a
training and test
caption to share the same article text (since articles have multiple images).





\begin{figure}[t]
   \begin{center}
   \includegraphics[width=0.99\linewidth]{figures/figure_2_entities.pdf}
   \end{center}
      \caption{Entity distribution in NYTimes800k training captions. The four
               most common entity types are personal names, geopolitical
               entities, organizations, and dates.}
   \label{fig:entities}
\end{figure}

\begin{figure}[t]
   \begin{center}
   \includegraphics[width=0.99\linewidth]{figures/figure_3_faces.pdf}
   \end{center}
      \caption{Co-occurrence of faces and personal names in NYTimes800k
               training data. The blue bars count how many images containing a
               certain number of faces. The orange bars count how many captions
               containing a certain number of personal names.}
   \label{fig:faces}
\end{figure}


\subsection{NYTimes800k}

We constructed the NYTimes800k which is an 80\% larger and more complete
dataset
of New York Times articles, images, and captions. The construction of this
dataset was motivated by the desire to clean up data quality issues in the
GoodNews dataset (as described below), collect a larger dataset, and include
fine grained context
such as the images location in the article.

We observed that many of the articles in the GoodNews dataset had
been partially extracted when the generic article extractor used
failed to recognise some of the HTML tags used
specifically by the New York Times. Importantly, the missing text often
included the first few
paragraphs
which frequently contain important information for captioning images. To
collect the full articles for the NYTimes800k dataset we
implemented a custom parser using The New York Times public
API\footnote{\href{https://developer.nytimes.com/apis}{https://developer.nytimes.com/apis}}.
After this recollection, we observe that the average article is 963 words in
comparison to GoodNews where the average length is 451 words. In addition, we
found that GoodNews contains a
small number of non-English articles, and captioned images from the
recommendation sidebar which are not related to the main article. These were
filtered out in the construction of NYTimes880k.


Table \ref{tab:datasets} presents a comparison between GoodNews and
NYTimes800k. NYTimes800k exhibits several advantages:

\begin{itemize}
   \item By increasing the collection period to the last 14 years (March 2005
   -- September 2019), NYTimes800k contains 80\% more articles and images, thus
   becoming the largest news image captioning dataset.
   \item Using our custom parser, articles in NYTimes800k contain the full text
   with no missing paragraphs.
   \item NYTimes800k contains only English articles.
   \item We are careful to include only images that are part of the main
   article.
   \item Unlike GoodNews, we also collect information about where an image is
   located in the corresponding article. Most news articles have one image at
   the top that relates to the key topic. However 39\% of the articles have at
   least one more image somewhere in the middle of text. The image placement
   and hence the text surrounding the image is important information for
   captioning as we show in our evaluations.
\end{itemize}

Entities play an important role in the dataset, with 97\% of captions
containing at least one named entity. As shown in Figure \ref{fig:entities},
the most popular entity type are names of people, comprising a third of all
named entities. Furthermore, 71\% of training images contain at least one face
and 68\% of training captions mention at least one persons name. Figure
\ref{fig:faces} provides a further breakdown of the co-occurrence of faces and
personal names. One important observation is that the majority of captions
contain at most four names.


\begin{table}[t]
	\caption {NYTimes800k training, validation, and test splits}
	\label{tab:splits}
	\centering
	\begin{tabularx}{\linewidth}{lXXX}
		\toprule
		  & Training  &   Validation & Test \\
		\midrule
      Number of articles & 434 272 & 3 052 & 8 495 \\
      Number of images  & 764 049 & 7 852 & 22 143 \\
      Start month & Mar 15 & May 19 & Jun 19 \\
      End month & Apr 19 & May 19 & Aug 19 \\
		\bottomrule
	\end{tabularx}
\end{table}

We split the training, validation, and test sets according to time, as shown in
Table \ref{tab:splits}. For example, the test set consists of all captions and
articles in the final three months of the collection period, from June to
August 2019. This has two advantages over the random split used in GoodNews.
Firstly, it prevents captions in the training and test sets from sharing the
same context article, which allows us to evaluate how well the model can
generalize to new articles. Secondly, due of the shift in the coverage of news
over time, there
will be events and people in the test data that have never been covered by the
news
before. In particular, out of the 100K proper nouns in the test captions, 4\%
never appear in any training captions. Half of these also never appear in any
training article. Thus splitting by time allows us study how well the model can
generate rare names.

% !TEX root = main.tex


\section{Experiments}

\subsection{Training Details}

In all experiments, we use Adam \cite{Kingma2015Adam} with the weight decay fix
\cite{Loshchilov2018DecoupledWD} to optimize the models. We use the following
parameters: $\beta_1 = 0.9, \beta_2 = 0.98, \epsilon = 10^{-6}$, and a weight
decay of $10^{-5}$. We clip the gradient norm at 0.1. All models see the same
number of training examples, which is 6.6 million. This is equivalent to 16
epochs through GoodNews and 9 epochs through NYTimes800k. We warm up the
learning rate in the first 5\% of the training steps to $10^{-4}$, and decays
it linearly afterwards. Training is done with mixed precision to reduce the
memory footprint. The full model takes 5 days to train on one Titan V GPU.

As shown in Table \ref{tab:models}, all of our baselines are designed to have
roughly the same number of trainable parameters. This allows us to attribute
improvement to the individual model components rather than to simply having a
bigger model.

The training pipeline is written in PyTorch \cite{Paszke2017Automatic} using
the AllenNLP framework \cite{Gardner2017AllenNLP}. The RoBERTa model and
dynamic convolution code are adapted from fairseq \cite{Ott2019Fairseq}.

\begin{table}[t]
	\caption {Model complexity}
	\label{tab:models}
	\centering
	\begin{tabularx}{\linewidth}{Xc}
		\toprule
        & No. of Parameters \\
      \midrule
      LSTM + GloVe & 157M \\
      Transformer + GloVe & 148M \\
      LSTM + weighted RoBERTa & 159M \\
      \midrule
      Transformer + RoBERTa & 154M \\
      \quad + weighted RoBERTa & 154M \\
      \quad\quad + face attention & 171M \\
      \quad\quad\quad + copying & 207M \\
		\bottomrule
	\end{tabularx}
\end{table}

\subsection{Results}

\begin{figure*}[t]
   \begin{center}
   \fbox{\rule{0pt}{3in} \rule{.9\linewidth}{0pt}}
   \end{center}
      \caption{Example captions with success and failure cases.}
   \label{fig:short}
\end{figure*}


Table \ref{tab:results} shows the BLEU \cite{Papineni2002Bleu}, ROUGE
\cite{Lin2004ROUGE}, METEOR \cite{Denkowski2014Meteor}, and CIDEr
\cite{Vedantam2015CIDEr} metrics on GoodNews and NYTimes800k. We show two best
results from the previous state-of-the-art \cite{Biten2019GoodNews}. Note that
the numbers reported here are slightly different from the original paper since
we had to remove a few samples from the test set where the image is no longer
available. \cite{Biten2019GoodNews} also did some post-processing on the
ground-truth captions such as removing contractions and non-ASCII characters,
both of which we did not do. Despite these differences, the final metrics
are the same if rounded to the nearest whole number.

There is a strong correlation between of all these metrics, and in general, we
mainly look at CIDEr since it uses Term Frequency Inverse Document Frequency
(TF-IDF) to put more importance on less common words such as entity names.
Table \ref{tab:names} shows the recall and precision of the named entities,
personal names, and rare proper nouns.

We can make the following observations:

\begin{itemize}
   \item Our baseline LSTM model with GloVe embeddings yields competitive
   results to previous the state-of-the-art \cite{Biten2019GoodNews}. This
   means that BPE offers a viable alternative to template-based methods.

   \item Models with GloVe embeddings are unable to generate rare proper nouns.
   This is expected since GloVe has a fixed vocabulary and if there is a
   unknown word in the article, the encoder will simply skip it.

   \item Switching from an LSTM to a transformer architecture improves the
   CIDEr score on NYTimes800k by 8 points, from 12 to 20. If we then use the
   contextualize RoBERTa embeddings instead of GloVe, CIDEr more than doubles
   to 44.

   \item Adding attention over the faces improves both the recall and precision
   of personal names. It has no significant effect on other entity types (see
   the supplementary materials for a detailed breakdown).
\end{itemize}


Table \ref{tab:names} also looks at the quality of the generated captions. We
look at three metrics: caption length, type-token ratio (TTR), and Flesch
reading ease. TTR is the ratio of the number of unique words to the total
number of words in a caption. The Flesch reading ease takes into account the
number of words and syllables and produces a score between 0 and 100, where
higher means being easier to read.

From these metrics, we see that our generated captions are in general still
shorter than real-life captions, have lower lexical diversity (lower TTR)
and still use simpler language (higher Flesch reading ease).

% !TEX root = main.tex
\section{Conclusion}


{\small
\bibliographystyle{ieee_fullname}
\bibliography{main}
}

\end{document}
